\chapter{Introduction (1-2 pages)}
The introduction chapter should not include detailed information on
how you intend to solve the problem, what you're going to do etc.
This belongs more in the 'method' and 'feasibility study' section
of the research proposal.

Make sure you read several of the past project proposals.
Make your own judgment on how 'good' they are.

\section{Topic covered by the project}
This section specifies the general area of the project.
It should preferably be understandable by everybody,
also those not familiar with the field. (e.g. all your relations and friends).

The purpose of the topic section is to:
\begin{itemize}
\item Very quickly give the reader some idea of the perspective taken
with respect to problem addressed.  
\item Help a reader to decide if the project
is within the readers area of interest and scope.
\item Help the author (you!) to see if he has the necessary skills,
if he/she needs to get access to specific expertise etc.
Do you have the right skills/ background/ knowledge
to be able to carry out the project?
\end{itemize}

\paragraph{Control questions:}
\begin{enumerate}
\item Does it have the right length?
\item Is it focused or is it just a non-focused brain dump going all over the place?
\item Is it clear from the text what skills would be required/beneficial in order to do/participate in the project?
\end{enumerate}


\section{Keywords}
There are several sources of keywords. Rather than 'inventing your own' you should select an appropriate set of keywords from a reputable source such as the one published by the IEEE computer society (IEEE Computer Society - Keywords) or ACM (ACM Computing Classification System).
%\protect\url{http://www.computer.org/portal/site/ieeecs/menuitem.c5efb9b8ade9096b8a9ca0108bcd45f3/index.jsp?&pName=ieeecs_level1&path=ieeecs/publications/author/keywords&file=ACMtaxonomy.xml&xsl=generic.xsl&},
The taxonomy by Avizienis et al\cite{avizienis2004} provides an overview of of the subject area and an alternative set of keywords/classifications.

\paragraph{Control questions:}
\begin{enumerate}
\item Does the collection of keywords 'pin down' the project or is it to 'wide'?
\item Are the keywords too specific, making it difficult  for people with a closely related interest to recognize the keywords?
\item Why is it likely that a person working in the field would use the keywords you have selected when doing a search in this area?
\item Is the number of keywords appropriate?
\end{enumerate}

\section{Problem description}
What's 'wrong' with the world we're living in? E.g.
\begin{itemize}
\item   Something is currently too difficult.
\item   Something is broken/doesn't work properly.
\item   Something is currently to expensive, difficult, costly etc.
\end{itemize}

\paragraph{Control questions:}
\begin{enumerate}
\item Does it have an appropriate length?
\item Would it be possible to explain the problem description to a non-expert/expert in say 2 minutes in such a way that it was understood?
\item If explained to different people, would they have a common understanding?
\item If you were to check if your problem description was understood, what question(s) would you ask?
\item What is the information density of your text and why?
\end{enumerate}

\section{Justification, motivation and benefits}
This section should be understandable by everybody including your family and relatives.
In particular, it should be understandable to those who will benefit.
NOTE : 'I want to do zz' does not count as a legitimate motivation!
\begin{itemize}
\item Why is important to solve the problem you have identified?
\item Why would 'mankind' benefit from a solution to the problem identified?
\item Who would benefit (the stakeholders)?
\item What are the primary and secondary benefits - what's in it for the stakeholders?
\end{itemize}
You should try to find a journal, conference or newspaper article identifying the problem you will be adressing.
This can be used to substantiate your claim that the problem you are adressing is significant.

\paragraph{Control questions}
\begin{enumerate}
\item For each of the issues listed above, has the issue been addressed properly/thoroughly? 
\item What is the information density of your text and why?
\item If the project results was to be put in an auction when the project was completed - what price would it fetch and who would put in what bids? 
\item What would be the overall ROI (Return On Investment) of your project if carried out?
\end{enumerate}

\section{Research questions}\label{research:questions}
Describe the types of information you need in order to solve the research problem, e.g.
We need to find out
\begin{itemize}
\item what factors affect  xx (where xx is the 'parameter' you want to improve, e.g. cost, time, usability, security, etc.)
\item to what extent will activity/ method/procedure yy (where yy is some method of improving the parameter, e.g.  a program for simplifying access) improve factor xx?
\item have somebody solved this or some closely related problem?
\item how well has the problem been solved?
\item what is the theoretically 'best' one can achieve?
\end{itemize}

\paragraph{Control questions:}
\begin{enumerate}
\item Are there any questions at all? Look for '?'...
\item Why are the research questions relevant to the research problem?
\item What other research questions might also be relevant?
\item why/why not are the chosen research questions the most relevant?
\end{enumerate}

\section{Planned contributions}
A short summary of what kind of {\em new} results the master thesis will produce.  
Ideally,  the potential novelty of the results should be justified by means of references provided.
E.g. if an article describes the problem you will be adressing as {\em unsolved},
you should include this reference.  Similarly, if you e.g. have some ideas on how an 
authentication method can be improved in terms of FAR/FRR, you should specify the best
 FAR/FRR figures published and a reference to where this was published. 
 The goal of the master thesis
will be to produce the new results identified in this section.

\paragraph{Control questions}
\begin{enumerate}
\item Is the length of the section appropriate and why?
\item Why/why not are the contributions 'significant'?
\item Why/why not is it realistic that the planned contribution can be achieved?  You may want to have a look at relevant literature/ other completed master thesis to answer this question.
\end{enumerate}
\chapter{Related work (3-10 pages)}
The purpose of this chapter
is to explain to the reader what knowledge is already
available from the literature.

The purpose of the related work chapter is to:
\begin{itemize}
\item Identify to what extent information identified in the 'Research questions'  section is provided in the literature.
\item Give an overview of why/how the literature provides the answer to the research questions identified.
\item Identify areas/ research questions where the literature appears to be weak or non-existent.
\end{itemize}
The Related Work Chapter is NOT:
\begin{itemize}
\item   A list of abstracts and summaries of more-or-less-relevant literature.
\end{itemize}
If you have
\begin{itemize}
\item   found some relevant literature
\item   made summaries of what you have written
\end{itemize}
you should
\begin{itemize}
\item reorganize these summaries to focus on the research questions you have identified.
\end{itemize}

This chapter should include one subsection for each of the research
questions identified in section \ref{research:questions}.  

\section{Handling Potential problems}
When searching for literature, you usually get too many hits or none at all...

\paragraph{Question 1} I don't find any relevant literature.

\paragraph{Answer 1.A}  Make a list of words, phrases, applications, abbreviations,
organizations, terminology etc. relevant for your area of interest.
Ask a librarian to sit with you for 20 minutes to formulate relevant
queries to available databases.  Record your findings.

\paragraph{Answer 1.B}  Go to the ACM (www.acm.org) or IEEE (www.ieee.org) web pages.
Identify the SIGs (Special Interest Groups) of these organizations.
Select the SIGs which looks the most interesting.
Most SIGs publish one or more journals and/or organize workshops or conferences.
Get hold of a few journals or proceedings and see if they're any interesting.


\paragraph{Question 2}  I've found a lot of papers.
They all look interesting, but I don't have time to read them all.

\paragraph{Answer 2.A}  Narrow your search.  Be more specific in your search.  Read the abstracts of the relevant articles before you read the full papers.

\paragraph{Answer 2B}  Find a citation index (e.g. \url{http://citeseer.ist.psu.edu/}.
Read those papers with a high citation score first
(a citation index rates papers according to 'academic popularity').  Alternatively,
read those papers published in 'prestigious' conference proceedings or journals first.


\paragraph{Control questions:}
\begin{enumerate}
\item Why can we have confidence that the most relevant literature has been identified?
\item is the related literature grouped in a sensible way such that the reader gets a good understanding of 'existing knowledge' relating to th research questions/problem description?
\item Is the chapter sufficiently comprehensive?
\end{enumerate}
\begin{abstract}
\if{false}
Abstract (1/2 page)
This document provides format and guidelines  for the 
MSc project descriptions. The document has been produced using MikTeX and TeXnicCenter.

The objective of the abstract is to provide the reader with an understanding of the work to be done and put him in the position to make a 'correct' decision regarding  reading/not reading the report.

The abstract of the project description
{\em must} include
\begin{itemize}
\item a summary of the problem description,
\item motivation and 
\item a summary of the planned contribution from the master project in terms of {\em new} results.
\end{itemize}

\paragraph{Control questions}
\begin{enumerate}
\item Does the abstract have a 'reasonable' length?
\item Is it clear to a non expert (e.g. a typical reader of a newspaper) what problem is addressed?
\item Does a person that has been working in the field find the text informative?
\item Do the results that might be obtained have the potential to be interesting to a lot of people? How interesting to how many and why?
\item Would a decision maker/manager be willing to pay NOK 400.000 to have the project completed (estimated salary costs + overheads) after having read the abstract?  Why/why not?
\end{enumerate}
\fi

The field of redirected walking aims to optimise the use of available physical space for room scale VR solutions by unnoticeably redirecting the user. By itself, redirected walking requires large rooms to perform at full effectiveness. This makes it hard to entirely redirect users in the current day consumer VR market as physical tracking space is limited. One way to mitigate this is through the use of advanced reorientation techniques like distractors that aim to keep the subjective sense of presence high compared to their alternatives. The field of distractors is generally quite small and as such, it would be beneficial to contribute to it. In particular, there does not seem to be much research in state of the art literature on noticeability and detection thresholds of redirected walking with distractors. The planned contribution for the master thesis is to provide an example of potential detection thresholds when making use of state of the art distractors as well as the effectiveness of redirection with this approach. 


\end{abstract}
\chapter{Risk Analysis}\label{chap:risk}
Throughout the project, there are a variety of risks that could impede progress. This chapter lists some of the most apparent risks and how to deal with these.

\if{false}
\begin{itemize}
\item What can possibly go wrong when you do your project?
\item How do you intend to reduce impact of/solve these problems?
\end{itemize}
\fi

\section{Illness}
The first risk is illness. This is generally something that can happen to anyone with varying degrees of strength. In order to mitigate any potential illness it would be preferable to work steadily every week and to try and avoid any major crunches to limit any burnout or stress related problems. Of course, if illness actually occurs, then the focus should be on acquiring enough rest so that a speedy recovery can take place. 

\section{Project Progress is Too Slow Relative to Necessary Time}
There could be times where project progress is moving along a bit too slowly relative to any weekly goals or remaining time. In order to improve progress it is possible to spend some additional time during the weekends. This should preferably not occur too often or with too much time spent during weekends as it is necessary to take some time to relax. In terms of development, the scope of the planned VR game is fairly modular. If needed, the game could be scoped down accordingly. Having a modular design with a minimum viable product as a main goal should help with this. 

\section{Supervisor Does Not Show up to Weekly Meeting}
Sometimes, the supervisor may be busy and not able to attend the weekly status meetings that are scheduled. This can be problematic if there are important things that need to be discussed. The first option to solve this would be to try and reschedule the weekly meeting through emails or other messaging platforms. Another option to mitigate this would be to have a second supervisor. This would result in some additional overhead every week in terms of status meetings, but could provide some additional perspective for discussions. Having a second supervisor can also work as a backup in cases where one of the supervisors are busy for the week. 

\section{There are Limited Amounts of Test Participants}
One major risk is potentially not finding enough test participants to measure accurate results. In order to find as many participants as possible, testing should preferably start early. Sending emails and asking groups of people does generally not result in a high respondent rate so it would be preferable to ask directly. A fair amount of time should be spent at least a week before the scheduled experiments to find participants. 

\section{There are no Rooms Available for Experiments}
Similarly as with participants, it can be hard to find a bookable room. Checking the availability of relevant experiment rooms and booking these should be done several weeks ahead of time. It is also possible to reserve some otherwise unavailable rooms by email which might be necessary for the bigger ones. 

\section{The Redirected Walking Toolkit Does Not Work With Later Versions of Unity}
Finally, it could be that the redirected walking toolkit by Azmandian et al.~\cite{azmandian2016redirected} does not work with later versions of Unity. In this case, it should be possible to update the source code to work with the newer versions as all of it is open source. Regardless, if it ends up being the case, some additional time will have to be spent on updating the toolkit. This work could be submitted as a pull request to the authors of the toolkit for the benefit of other redirected walking researchers and developers.  

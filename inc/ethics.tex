\chapter{Ethical and Legal Considerations} \label{chap:ethics}

\if{false}
The purpose of this chapter is to convince the
reader and your self that your project activities are
\begin{itemize}
\item legal
\item ethical, e.g. don't use/distribute/collect etc. data in such a way that individuals may suffer.
\end{itemize}
For example, if you are planning to do reverse engineering activities, surveys, PENTESTING- (both technical and based on social engineering techniques) you need to be particularly careful and check with the appropriate experts and authorities if the activity is permitted.  An explanation of why your project is both legal and ethical should be given in this chapter.

If you need permission (e.g. because you will be collecting or processing privacy related information), 
you should  include the appropriate applications/application forms and ensure that these applications 
are submitted well before you need the permission.
\fi

As part of completing the thesis, it is necessary to consider the various ethical ramifications of the experiments as well as any legal necessities for data collection.

\section{Ethics of Researching Redirected Walking}
One problem with researching redirected walking is the ethics around potentially making participants cybersick. This can be particularly problematic when estimating detection thresholds with standard methods like Steinicke et al. have used~\cite{5072212}. Their method for estimating detection thresholds makes used of a two-alternative forced choice task with a uniform distribution of gains in a given range. These gains are then tested in random order where the participant has to answer whether the current gains are higher or lower than the norm. The problem with this approach is that this range of gains could have values that are high or low enough to result in cybersickness. As such, it could be considered to be unethical to estimate detection thresholds like this. Instead, the plan is to try an alternative method which is mentioned in Chapter~\ref{chap:methods}. 

Another thing that should be noted is that participants will be told to stop the experiment at once if they experience any cybersickness in general. While prematurely ending the experiment with one participant does not provide fully estimated data regarding detection thresholds, it can still hold value in terms of trying to understand why the cybersickness occurred. 

\section{Informed Consent and Permissions for Data Collection}
As part of making the experiments legal, it is necessary to have informed consent from participants as well as formal permission from NSD\footnote{\url{http://www.nsd.uib.no/personvernombud/}} to collect data. The participants will be informed about everything the study collects and the procedure they will go through. A small pre-test questionnaire will be used for the sake of demographics data, but the data collection in general will stay anonymous.  
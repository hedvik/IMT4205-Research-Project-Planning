\chapter{Feasibility Study}
\if{false}
An analysis of why it is likely that the desired
results can be produced within the given time and
resource bounds.  This may include a description of
\begin{itemize}
\item similar projects completed by others and their 'resource consumption',
\item an attempt to answer parts of the research questions
\item the 'difficult' elements of the work and an explanation of why/how these problems can be solved.  
Alternatively you can explain an 'approximate' solution.
\end{itemize}
\fi

One crucial part of the planning is to discuss how feasible it is to complete the thesis. This short chapter is focused around this.


In general, the completion of the thesis should be feasible for a variety of reasons. First of all, I have previous experience with game programming and five years of experience with the Unity game engine. As such, the game development side of the thesis should not be too problematic. The game itself will have a reasonably simple scope, but with a modular design that can be expanded if additional time allows. The most challenging part here would be the concept of designing a game around distractors. Overall though, it should be feasible to create something that the participants can find engaging during the experiments.

The most challenging part in general will be to find a large enough amount of participants so that the estimated detection thresholds from experiment 1 and the results from experiment 2 are accurate enough. This is listed as a risk in Chapter~\ref{chap:risk} with respective measures for how the challenge should be taken care of. I expect that experiment 1 will be harder to find participants for as it requires more time compared to experiment 2. Despite this, it should be possible to find a minimum of \textasciitilde 15 participants for experiment 1 which would be similar to that of Steinicke et al.'s study which consisted of 14 participants~\cite{5072212}. One thing to note is that their study spent 3 hours per participant, but it would be unfeasible to spend similar amounts of time here. 

When it comes to similar projects, Fuglestad has also performed a redirected walking project of similar scope for his master thesis~\cite{fuglestad2018redirected}. As such, this project should also be feasible to complete.
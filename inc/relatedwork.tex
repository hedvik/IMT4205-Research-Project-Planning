\chapter{Related work (3-10 pages)}

\section{Redirected Walking, Detection Thresholds and Comfort}
* Slight background first

* Generally, most studies have done what they can to sustain the illusion of unconstrained walking in VR
* As such Steinicke et al. formalised the concept of detection thresholds which are thresholds where the user notices/detects that they are being redirected
* There are quite a few factors and variables that can affect the effectiveness of RDW and detection thresholds
   * Discuss 
* How much tracking space is feasible?
   * 22m ideally according to Steinicke et al.
   * The more gains you can apply the better
   * Inattentional blindness
   * Distractors
* Effects on comfort and cybersickness in relation to gain strength and detection
   * One of the benefits with natural walking is that it minimizes cybersickness compared to other locomotion techniques
   * Ideally you would want to limit the extra amount of cybersickness that comes as a result of redirection, otherwise there isn't that much point in using it other than improving immersion 
   * Dynamic FoV can make it harder to detect redirection, but it might also result in more sickness
* Immersion and sense of presence

* TOTHINK:
    * How does the literature provide answers to my research questions?
    * How can I apply what the literature has found to my project?

\section{Taxonomy of Distractors in Redirected Walking}
* What is a distractor?
   * A somewhat encompassing term
   * Some history: who introduced it to RDW? Peck et al. 
   * At the simplest level, a distractor when employed with redirected walking aims at occupying the user's attention in a way that makes it harder to notice that higher gains are being applied. 
      * Research points towards this phenomenon: citations
   * There is no direct taxonomy that defines what types of distractors exist
   * As such, a grounded theory approach has been used to generate a taxonomy of distractors which will be used as a framework for the following section in this literature review. 
   
* What are the dimensions we are dealing with?
   * General Traits
       * Taking a look at attentional cues from Cinematic VR: Explicit vs. implicit
           * If the "distractor" is very important in the VE, is it explicit?
              * Should an angry dragon be considered explicit?
              * Watching birds is explicit if that is your goal
              * Does it fill an explicit goal?
                 * If this goal is provided to the user, I would say that the distractor can be considered as explicit
           * Is a activity in a VE considered as a explicit distractor?
               * Might need some thinking!
           * An explicit one could be better at making the user turn around
           * Are implicit ones less effective?
               * Unless they are particularly salient, probably.
               * It might not be ideal to use implicit distractors
               * It can be dangerous to rely on the fact that the user will not ignore them
               * Large amounts of implicit distractors might create too much optical flow which may or may not make redirection more noticeable. 
           * Is the distractor something you can ignore, or is it something you really should keep attention to?
               * Ignoring a implicit distractor is probably easier than a explicit one
       * Context sensitivity
          * A generic distractor would be closer to standard reorientation techniques.
             * Not ideal in terms of presence or immersion, can become too repetitive
             * Looking at the literature, these have also generally required additional instruction before conducting the experiment
             * A generic distractor is not necessarily an important part of the experience
          * A context sensitive distractor is directly integrated into the virtual experience. 
             * They are a integral part of the experience.
             * Generally a good fit with games
       * Timing
            * When does the distractor activate?
            * Safe circles from previous research
            * Generally this happens earlier compared to a regular reorientation technique
   
   * Distractor Types
      * Is the distractor an object or a virtual existence in the experience?
         * Diegetic vs. Non-diegetic
            * Sense of presence and immersion
         * Is it static or is it moving?
             * How does it move? 
                * Towards or away from the user?
                * Moving around in the user's vision?
                * Circling the user?
         * Where does it appear in the user's vision?
            * In the peripheral vision?
            * In plain sight?
            * Outside of your sight?
               * Audio cues
               * Visual effects that signify that something is behind you or close to you
         * Is this something you want to avoid or is it something you want to engage with?
            * Deterrent, attractor
            * Deterrents can be used as a "chaperone" system similar to what you see with the HTC Vive
         * Salience
            * How quickly does it catch the user's attention?
            * How long can it keep the user's attention?
      * Is the distractor more abstract?
         * Is it an activity?
            * This will "attract" the user's attention
         * Does this activity consist of other "distractors"?
      
* In cinematic VR, there generally is a focus on implicit ways to guide attention for the sake of immersion in the experience.
* In games, the usage of explicit cues/distractors can make a bit more sense due to the more interactive nature of the environment. 
* TOTHINK: What elements of these define their own categories? Which ones are the most important?


\section{Redirected Walking with Distractors}
* Slight background first

* TOTHINK:
    * How does the literature provide answers to my research questions?
    * How can I apply what the literature has found to my project?
    * What is the state of the art lacking?
    * Use framework to see what types of distractors have been used
       * Discuss how in games, we can probably make use of more nondiegetic distractors without it affecting too much due to suspension of disbelief
    * Does any of the recent studies with distractors look at cybersickness?


The purpose of this chapter
is to explain to the reader what knowledge is already
available from the literature.

The purpose of the related work chapter is to:
\begin{itemize}
\item Identify to what extent information identified in the 'Research questions'  section is provided in the literature.
\item Give an overview of why/how the literature provides the answer to the research questions identified.
\item Identify areas/ research questions where the literature appears to be weak or non-existent.
\end{itemize}
The Related Work Chapter is NOT:
\begin{itemize}
\item   A list of abstracts and summaries of more-or-less-relevant literature.
\end{itemize}
If you have
\begin{itemize}
\item   found some relevant literature
\item   made summaries of what you have written
\end{itemize}
you should
\begin{itemize}
\item reorganize these summaries to focus on the research questions you have identified.
\end{itemize}

This chapter should include one subsection for each of the research
questions identified in section \ref{research:questions}.  



\section{Handling Potential problems}
When searching for literature, you usually get too many hits or none at all...

\paragraph{Question 1} I don't find any relevant literature.

\paragraph{Answer 1.A}  Make a list of words, phrases, applications, abbreviations,
organizations, terminology etc. relevant for your area of interest.
Ask a librarian to sit with you for 20 minutes to formulate relevant
queries to available databases.  Record your findings.

\paragraph{Answer 1.B}  Go to the ACM (www.acm.org) or IEEE (www.ieee.org) web pages.
Identify the SIGs (Special Interest Groups) of these organizations.
Select the SIGs which looks the most interesting.
Most SIGs publish one or more journals and/or organize workshops or conferences.
Get hold of a few journals or proceedings and see if they're any interesting.


\paragraph{Question 2}  I've found a lot of papers.
They all look interesting, but I don't have time to read them all.

\paragraph{Answer 2.A}  Narrow your search.  Be more specific in your search.  Read the abstracts of the relevant articles before you read the full papers.

\paragraph{Answer 2B}  Find a citation index (e.g. \url{http://citeseer.ist.psu.edu/}.
Read those papers with a high citation score first
(a citation index rates papers according to 'academic popularity').  Alternatively,
read those papers published in 'prestigious' conference proceedings or journals first.


\paragraph{Control questions:}
\begin{enumerate}
\item Why can we have confidence that the most relevant literature has been identified?
\item is the related literature grouped in a sensible way such that the reader gets a good understanding of 'existing knowledge' relating to th research questions/problem description?
\item Is the chapter sufficiently comprehensive?
\end{enumerate}